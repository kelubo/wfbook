% 中文书籍LaTeX模板
% 使用xelatex编译

\documentclass[12pt,a4paper,twoside]{ctexbook}

% 页面设置
\usepackage{geometry}
\geometry{a4paper, margin=1in}

% 字体设置
\usepackage{xeCJK}
\usepackage{fontspec}
\usepackage{microtype}

% 设置中文字体
\setCJKmainfont{SimSun}[  % 正文宋体
    BoldFont=SimHei,        % 粗体黑体
    ItalicFont=KaiTi        % 斜体楷体
]
\setCJKsansfont{SimHei}    % 无衬线字体黑体
\setCJKmonofont{SimSun}    % 等宽字体宋体
\setCJKfamilyfont{kai}[    % 楷体
    BoldFont=KaiTi
]{KaiTi}
\setCJKfamilyfont{fs}[     % 仿宋
    BoldFont=FangSong
]{FangSong}

% 常用字体命令
\newcommand{\song}{\CJKfamily{zhsong}}
\newcommand{\hei}{\CJKfamily{zhhei}}
\newcommand{\kai}{\CJKfamily{kai}}
\newcommand{\fs}{\CJKfamily{fs}}

% 标题格式设置
\ctexset{
    part/name={第,卷},
    part/number={\chinese{part}},
    chapter/name={第,章},
    chapter/number={\chinese{chapter}},
    section/name={第,节},
    section/number={\arabic{section}},
    subsection/number={\arabic{section}.\arabic{subsection}},
    chapter/format={\centering\hei\zihao{2}},
    section/format={\hei\zihao{4}},
    subsection/format={\hei\zihao{5}}
}

% 目录设置
\usepackage{titletoc}
\titlecontents{chapter}[0pt]{\vspace{10pt}\bfseries\zihao{-4}}{\contentspush{\thecontentslabel\hspace{1em}}}{}{\titlerule*[8pt]{.}\contentspage}
\titlecontents{section}[2.5em]{\vspace{5pt}\zihao{5}}{\contentspush{\thecontentslabel\hspace{1em}}}{}{\titlerule*[8pt]{.}\contentspage}
\titlecontents{subsection}[5em]{\vspace{3pt}\zihao{5}}{\contentspush{\thecontentslabel\hspace{1em}}}{}{\titlerule*[8pt]{.}\contentspage}

% 页眉页脚设置
\usepackage{fancyhdr}
\pagestyle{fancy}
\fancyhf{}
\fancyhead[LE,RO]{\zihao{5}\thepage}
\fancyhead[LO]{\zihao{5}\leftmark}
\fancyhead[RE]{\zihao{5}\rightmark}
\renewcommand{\chaptermark}[1]{\markboth{\chaptername\ \thechapter\ #1}{}}
\renewcommand{\sectionmark}[1]{\markright{\thesection\ #1}}
\fancyfoot[C]{\zihao{5} \thepage}
\renewcommand{\headrulewidth}{0.4pt}
\renewcommand{\footrulewidth}{0pt}

% 插图设置
\usepackage{graphicx}
\usepackage{float}
\usepackage{subfigure}
\graphicspath{{Images/}}
\floatstyle{plaintop}
\restylefloat{figure}

% 表格设置
\usepackage{tabularx}
\usepackage{booktabs}
\usepackage{longtable}

% 数学公式设置
\usepackage{amsmath, amssymb, amsthm}
\usepackage{mathrsfs}

% 定理环境
\newtheorem{theorem}{定理}[chapter]
\newtheorem{definition}{定义}[chapter]
\newtheorem{lemma}{引理}[chapter]
\newtheorem{corollary}{推论}[chapter]
\newtheorem{example}{例}[chapter]

% 目录、摘要等设置
\usepackage{makeidx}
\makeindex

% 摘要设置
\newenvironment{abstract}{
    \cleardoublepage
    \thispagestyle{empty}
    \begin{center}
        \textbf{\zihao{1} 摘要}
    \end{center}
    \vspace{1cm}
    \itshape
}{
    \normalfont
}

% 关键词设置
\newcommand{\keywords}[1]{
    \vspace{1cm}
    \noindent\textbf{关键词:} #1
}

% 引用设置
\usepackage{hyperref}
\hypersetup{
    colorlinks=true,
    linkcolor=blue,
    citecolor=blue,
    urlcolor=blue,
    pdftitle={Linux操作系统},
    pdfauthor={作者名},
    pdfsubject={Linux操作系统教程},
    pdfkeywords={Linux, 操作系统, RHEL, 内核, 发行版}
}

% 目录深度
\setcounter{tocdepth}{3}
\setcounter{secnumdepth}{3}

% 标题页设置
\usepackage{titling}

% 封面信息
\title{\hei\zihao{0} Linux操作系统}
\author{\song\zihao{2} 作者名}
\date{\song\zihao{4} \today}

\begin{document}

% 封面
\begin{titlepage}
    \begin{center}
        \vspace*{6cm}
        \hei\zihao{0} Linux操作系统
        \vspace*{3cm}
        \song\zihao{2} 作者名
        \vspace*{3cm}
        \song\zihao{4} \today
    \end{center}
\end{titlepage}

% 版权页
\newpage
\thispagestyle{empty}
\begin{center}
    \vspace*{8cm}
    \song\zihao{5} 版权所有 \textcopyright\ 2026 作者名
    \vspace*{1cm}
    \song\zihao{5} 出版社名称
\end{center}

% 摘要
\begin{abstract}
Linux是一个自由、开放的操作系统,具有高效、安全、稳定等特点,支持多种硬件平台,用户界面友好,网络功能强大,支持多任务、多用户。本书全面介绍了Linux操作系统的历史与发展、体系结构、版本分类、安装配置、命令行操作、用户权限管理、软件包管理、系统管理、网络配置、脚本编程以及高级应用等内容,帮助读者系统地掌握Linux操作系统的使用和管理技能。
    
    \keywords{Linux \quad 操作系统 \quad RHEL \quad 内核 \quad 发行版}
\end{abstract}

% 目录
\newpage
\tableofcontents

% 正文开始
\mainmatter

\chapter{Linux概述}

\section{Linux的历史与发展}

Linux操作系统是一个类似UNIX的操作系统。Linux操作系统是UNIX在计算机上的完整实现,它的标志是一个名为Tux的可爱的小企鹅形象。UNIX操作系统是1969年由肯·莱恩·汤普森(Kenneth Lane Thompson)和丹尼斯·里奇(Dennis Ritchie)在美国贝尔实验室开发的一个操作系统。由于良好且稳定的性能,该操作系统迅速在计算机中得到广泛应用,在随后的几十年中又不断地被改进。

\begin{figure}[H]
    \centering
    \includegraphics[width=0.8\linewidth]{linux-tux.png}
    \caption{Linux标志Tux小企鹅}
    \label{fig:tux}
\end{figure}

1990年,芬兰人莱纳斯·贝内迪克特·托瓦尔兹(Linus Benedict Torvalds)接触了为教学而设计的Minix系统后,开始着手研究编写一个开放的、与Minix系统兼容的操作系统。1991年10月5日,莱纳斯在芬兰赫尔辛基大学的一台FTP服务器上发布了一个消息。这也标志着Linux操作系统诞生。莱纳斯公布了第一个Linux的内核0.02版本。开始,莱纳斯的兴趣在于了解操作系统的运行原理,因此Linux早期的版本并没有考虑最终用户的使用,只是提供了最核心的框架,使得Linux开发人员可以享受编制内核的乐趣,但这样也保证了Linux操作系统内核的强大与稳定。互联网(Internet)的兴起,使得Linux操作系统也十分迅速地发展,很快就有许多程序员加入Linux操作系统的编写行列。

随着编程小组的扩大和完整的操作系统基础软件的出现,Linux开发人员认识到,Linux已经逐渐变成一个成熟的操作系统。1994年3月,内核1.0版本的推出,标志着Linux第一个正式版本诞生。

\section{Linux的版权问题}
Linux是基于Copyleft(无版权)的软件模式进行发布的。其实Copyleft是与Copyright(版权所有)相对立的新名称,它是GNU项目制定的通用公共许可证(General Public License,GPL)。GNU项目是由理查德·斯托尔曼(Richard Stallman)于1984年提出的。他建立了自由软件基金会(Free Software Foundation,FSF),并提出GNU计划的目的是开发一个完全自由的、与UNIX类似但功能更强大的操作系统,以便为所有的计算机用户提供一个功能齐全、性能良好的基本系统。

\section{Linux的特点与优势}
Linux操作系统作为一个自由、开放的操作系统,其发展势不可当。它拥有高效、安全、稳定,支持多种硬件平台,用户界面友好,网络功能强大,以及支持多任务、多用户等特点。

\begin{itemize}
    \item 开源与自由软件精神
    \item 稳定性与可靠性
    \item 安全性
    \item 高性能与高并发处理能力
    \item 灵活性与可定制性
    \item 广泛的硬件支持
\end{itemize}

\section{Linux的应用领域}
\begin{itemize}
    \item 服务器领域
    \item 嵌入式系统
    \item 云计算与大数据
    \item 移动设备(Android基于Linux内核)
    \item 桌面应用
    \item 科学计算与人工智能
\end{itemize}

\section{Linux的体系结构}
Linux一般由3个部分组成:内核(Kernel)、命令解释层(shell或其他操作环境)、实用工具。

内核是系统的"心脏",是运行程序、管理磁盘及打印机等硬件设备的核心程序。命令解释层向用户提供一个操作界面,从用户那里接受命令,并且把命令送给内核去执行。由于内核提供的都是操作系统最基本的功能,所以如果内核发生问题,那么整个计算机系统就可能会崩溃。

shell是系统的用户界面,提供用户与内核进行交互操作的接口。它接收用户输入的命令,并且将命令送入内核去执行。

命令解释层在操作系统内核与用户之间提供操作界面,可以称其为一个解释器。操作系统对用户输入的命令进行解释,再将其发送到内核。Linux存在几种操作环境,分别是桌面(desktop)、窗口管理器(window manager)和命令行shell(command line shell)。Linux操作系统中的每个用户都可以拥有自己的用户操作界面,即根据自己的需求进行定制。

shell也是一个命令解释器,解释由用户输入的命令,并把命令送到内核。不仅如此,shell还有自己的编程语言,可用于命令的编辑,它允许用户编写由shell命令组成的程序。shell编程语言具有普通编程语言的很多特点,如它也有循环结构和分支控制结构等。用这种编程语言编写的shell程序与其他应用程序具有同样的效果。

\subsection{shell提示符}
shell提示符是shell向用户显示的命令输入提示,通常显示在命令行的开头。不同的Linux发行版和shell类型可能会有不同的提示符格式,但它们通常包含以下信息:

\begin{itemize}
    \item \textbf{用户名}:当前登录的用户名称
    \item \textbf{主机名}:系统的主机名称
    \item \textbf{当前目录}:用户当前所在的工作目录
    \item \textbf{权限级别}:通常用不同的符号表示,如root用户用\#,普通用户用\$
\end{itemize}

常见的shell提示符格式:
\begin{verbatim}
# root用户提示符示例
[root@localhost ~]#

# 普通用户提示符示例
[user@localhost ~]$
\end{verbatim}

其中:
\begin{itemize}
    \item \verb|root| 或 \verb|user| 是用户名
    \item \verb|localhost| 是主机名
    \item \verb|~| 表示用户的主目录
    \item \verb|#| 表示root用户,\verb|$| 表示普通用户
\end{itemize}

用户可以通过修改shell配置文件(如\verb|~/.bashrc|)来自定义提示符的格式,使其显示更多或更简洁的信息。

标准的Linux操作系统都有一套叫作实用工具的程序,它们是专门的程序,如编辑器、执行标准的计算操作等。用户也可以使用自己的工具。

实用工具可分为以下3类:
\begin{itemize}
    \item 编辑器:用于编辑文件
    \item 过滤器:用于接收数据并过滤数据
    \item 交互程序:允许用户发送信息或接收来自其他用户的信息
\end{itemize}

\section{Linux的版本}
Linux的版本分为内核版本和发行版本两种。

\subsection{内核版本}
内核是系统的"心脏",是运行程序、管理磁盘及打印机等硬件设备的核心程序,提供了一个在裸设备与应用程序间的抽象层。例如,程序本身不需要了解用户的主板芯片集或磁盘控制器的细节就能在高层次上读/写磁盘。

内核的开发和规范一直由莱纳斯领导的开发小组控制着,版本也是唯一的。开发小组每隔一段时间公布新的版本或其修订版,从1991年10月莱纳斯向世界公开发布的内核0.0.2版本(0.0.1版本功能相当"简陋",所以没有公开发布),到目前最新的内核5.10.12版本,Linux的功能越来越强大。

Linux内核的版本号命名是有一定规则的,版本号的格式通常为"主版本号.次版本号.修正号"。主版本号和次版本号标志着重要的功能变更,修正号表示较小的功能变更。以2.6.12为例,2代表主版本号,6代表次版本号,12代表修正号。

可以到Linux内核官方网站下载最新的内核代码。

\subsection{发行版本}
仅有内核而没有应用软件的操作系统是无法使用的,所以许多公司或社团将内核、源代码及相关的应用程序组织构成一个完整的操作系统,让一般的用户可以简便地安装和使用Linux,这就是所谓的发行版(Distribution)。一般谈论的Linux操作系统便是针对这些发行版的。目前各种发行版超过300种,它们的发行版本号各不相同,使用的内核版本号也可能不一样,现在流行的Linux操作系统套件有RHEL、CentOS、Fedora、openSUSE、Debian、Ubuntu等。

\subsubsection{RHEL 8}
作为面向云环境和企业IT的强大企业级Linux操作系统,RHEL8版本于2019年5月8日发布。在RHEL 7系列发布约5年之后,RHEL 8在优化诸多核心组件的同时引入了诸多强大的新功能,支持各种工作负载,从而可以让用户轻松驾驭各种环境。

RHEL 8为"混合云时代"的到来引入了大量新功能,包括用于配置、管理和修复RHEL 8的Red Hat Smart Management扩展程序,以及包含快速迁移框架、编程语言和诸多开发者工具在内的Application Streams。

RHEL 8同时对管理员和管理区域进行了改善,让系统管理员、Windows管理员更容易访问。此外,通过Red Hat Enterprise Linux System Roles,Linux初学者可以更快地自动化执行复杂任务,以及通过RHEL Web控制台管理和监控RHEL的运行状况。

在安全方面,RHEL 8内置了对OpenSSL 1.1.1和TLS 1.3加密标准的支持。它还为Red Hat容器工具包提供全面的支持,用于创建、运行和共享容器化应用程序,改进对ARM和POWER架构、SAP解决方案和实时应用程序,以及Red Hat混合云基础架构的支持。

\subsubsection{BaseOS和AppStream}
RHEL 8提出了一个新的设计理念,将软件仓库分为BaseOS和AppStream两部分:

\paragraph{BaseOS}
以传统RPM软件包的形式提供操作系统底层软件的核心集,是基础软件安装库。BaseOS提供核心操作系统功能,包括系统的基础组件和底层服务。这些软件包具有较长的生命周期,与RHEL版本的生命周期一致,确保系统的稳定性和可靠性。BaseOS中的软件包通常不会频繁更新,主要提供安全补丁和错误修复。

\paragraph{AppStream(应用程序流)}
包括额外的用户空间应用程序、运行时语言和数据库,以支持不同的工作负载和用例。AppStream中的内容有两种格式:
\begin{itemize}
    \item 熟悉的传统RPM格式
    \item 称为模块(Module)的RPM格式扩展
\end{itemize}

AppStream采用模块化设计,允许在独立的生命周期中安装其他版本的软件,同时保留核心操作系统软件包。这使用户能够安装同一个程序的多个主要版本,例如同时安装不同版本的Python或Node.js。

通过AppStream,用户可以更轻松地升级用户空间软件包,而不会影响核心操作系统的稳定性。AppStream中的软件包通常有更频繁的更新周期,以提供最新的功能和改进。

这种分离设计使得RHEL 8能够更好地平衡系统稳定性和应用程序创新性,满足现代企业环境中不同工作负载的需求。

\chapter{Linux系统安装与配置}

\section{系统安装前的准备}
\begin{itemize}
    \item 硬件要求与兼容性检查
    \item 发行版选择建议
    \item 安装介质制作(U盘/光盘)
    \item 分区规划原则
\end{itemize}

在启动RHEL 8安装程序前,需根据实际情况的不同,准备RHEL 8 DVD安装映像,同时要进行分区规划。

对于初次接触Linux的用户来说,分区方案越简单越好,所以最好的选择就是为Linux准备3个分区,即用户保存系统和数据的根分区(/)、启动分区(/boot)和交换分区(swap)。其中,交换分区不用太大,与物理内存同样大小即可;启动分区用于保存系统启动时所需要的文件,一般500MB就够了;根分区则需要根据Linux操作系统安装后占用资源的大小和所需要保存数据的多少来调整大小(一般情况下,划分15GB~20GB就足够了)。

特别注意:如果选择的固件类型为"UEFI",则Linux操作系统至少必须建立4个分区:根分区、启动分区、EFI启动分区(/boot/efi)和交换分区。

当然,对于"Linux熟手",或者要安装服务器的管理员来说,这种分区方案就不太适合了。此时,一般会再创建一个/usr分区,操作系统基本都在这个分区中;还需要创建一个/home分区,所有的用户信息都在这个分区下;还有/var分区,服务器的登录文件、邮件、Web服务器的数据文件都会放在这个分区中。

\section{Linux安装过程}
任何硬盘在使用前都要进行分区。硬盘的分区有两种类型:主分区和扩展分区。RHEL 8提供了多达4种安装方式支持,可以从CD-ROM/DVD启动安装、从硬盘安装、从NFS服务器安装或者从FTP/HTTP服务器安装。

可扩展固件接口(Unified Extensible Firmware Interface,UEFI)启动需要一个独立的分区,它将系统启动文件和操作系统本身隔离,可以更好地保护系统的启动。

UEFI启动方式支持的硬盘容量更大。传统的基本输入输出系统(Basic Input Output System,BIOS)启动由于受主引导记录(Master Boot Record,MBR)的限制,默认无法引导2.1TB以上的硬盘。随着硬盘价格的不断下降,2.1TB以上的硬盘会逐渐普及,因此UEFI启动也是今后主流的启动方式。

\begin{itemize}
    \item BIOS/UEFI设置
    \item 图形化安装界面操作
    \item 分区方案(MBR/GPT,根分区,swap分区,/home分区等)
    \item 系统引导程序配置(GRUB)
    \item 用户设置与网络配置
\end{itemize}

\section{系统初始化与基本配置}
\begin{itemize}
    \item 首次启动与登录
    \item 网络配置(静态IP/动态IP)
    \item 系统更新与软件源配置
    \item 语言与时区设置
    \item 基本硬件驱动安装
\end{itemize}

\subsection{Linux开机过程}
Linux操作系统的开机过程按以下步骤进行:

\begin{enumerate}
    \item \textbf{BIOS/UEFI初始化}:计算机启动后,首先执行BIOS(基本输入输出系统)或UEFI(统一可扩展固件接口)的自检,检测硬件设备并确定启动设备顺序。
    \item \textbf{Boot Loader启动}:从启动设备加载引导加载程序(如GRUB),引导加载程序负责加载Linux内核。
    \item \textbf{内核加载}:引导加载程序将内核镜像加载到内存并执行,内核开始初始化系统硬件。
    \item \textbf{内核初始化}:内核检测并初始化各种硬件设备,挂载根文件系统,准备运行用户空间程序。
    \item \textbf{启动初始化进程}:内核启动第一个用户空间进程(PID为1),即初始化进程,负责完成系统的初始化工作。
\end{enumerate}

\subsection{systemd初始化进程}
初始化进程作为Linux操作系统的第一个进程,需要完成Linux操作系统中相关的初始化工作,为用户提供合适的工作环境。

RHEL 8已经替换了传统的System V init初始化进程服务,正式采用全新的systemd初始化进程服务。systemd具有以下特点:

\begin{itemize}
    \item \textbf{并发启动机制}:systemd采用并行启动服务的方式,大大缩短了开机时间。
    \item \textbf{统一的服务管理}:使用systemctl命令管理系统服务,替代了传统的service和chkconfig命令。
    \item \textbf{依赖关系管理}:自动处理服务间的依赖关系,确保服务按正确顺序启动。
    \item \textbf{按需启动}:支持服务的按需启动,提高系统资源利用率。
    \item \textbf{日志管理}:集成了journald日志系统,提供统一的日志管理。
\end{itemize}

通过这些改进,systemd显著提升了Linux系统的开机速度和管理效率,成为现代Linux发行版的标准初始化系统。

\paragraph{目标(Target)替代运行级别}
RHEL 8采用systemd后,不再使用传统的"运行级别"概念。Linux操作系统在启动时要进行大量的初始化工作,如挂载文件系统、交换分区和启动各类进程服务等,这些都被视为一个一个的单元(Unit)。

systemd用目标(Target)代替了System V init中运行级别的概念。目标是一组相关单元的集合,用于实现特定的系统状态。例如:
\begin{itemize}
    \item \textbf{multi-user.target}:对应传统的运行级别3,提供多用户命令行界面
    \item \textbf{graphical.target}:对应传统的运行级别5,提供图形用户界面
    \item \textbf{rescue.target}:对应传统的单用户模式,用于系统救援
    \item \textbf{poweroff.target}:关机状态
    \item \textbf{reboot.target}:重启状态
\end{itemize}

这种基于目标的设计更加灵活和可扩展,能够更好地适应现代Linux系统的需求。

\subsubsection{启动级别与Target对应关系}
System V init的运行级别与systemd的target之间存在以下对应关系:

\begin{table}[htbp]
    \centering
    \begin{tabularx}{\textwidth}{|X|X|X|}
        \hline
        \textbf{System V运行级别} & \textbf{描述} & \textbf{对应的systemd目标} \\
        \hline
        \textbf{0} & 关机 & poweroff.target \\
        \hline
        \textbf{1} & 单用户模式 & rescue.target \\
        \hline
        \textbf{2} & 多用户模式(无网络) & multi-user.target \\
        \hline
        \textbf{3} & 多用户模式(有网络) & multi-user.target \\
        \hline
        \textbf{4} & 未使用 & 无直接对应 \\
        \hline
        \textbf{5} & 图形界面模式 & graphical.target \\
        \hline
        \textbf{6} & 重启 & reboot.target \\
        \hline
        \textbf{无对应运行级别} & 紧急模式(最基础的救援模式) & emergency.target \\
        \hline
    \end{tabularx}
    \caption{System V init运行级别与systemd目标对应关系}
    \label{tab:runlevel_target_mapping}
\end{table}

\paragraph{关于emergency.target}
emergency.target是systemd中的紧急救援模式,它提供了最基础的系统访问环境,比rescue.target更加基础,通常用于解决严重的系统问题。

在emergency.target模式下:
\begin{itemize}
    \item 系统只挂载根文件系统为只读模式
    \item 不启动任何网络服务
    \item 不启动任何其他系统服务
    \item 只提供最基本的shell环境
    \item 适合处理严重的文件系统损坏、引导配置错误等问题
\end{itemize}

相比之下,rescue.target会挂载所有本地文件系统,启动一些基本服务,提供更完整的救援环境。

\subsection{systemd与System V init的区别}
\begin{table}[htbp]
    \centering
    \begin{tabularx}{\textwidth}{|X|X|X|}
        \hline
        \textbf{特性} & \textbf{System V init} & \textbf{systemd} \\
        \hline
        \textbf{启动方式} & 串行启动,服务按顺序启动 & 并行启动,服务同时启动 \\
        \hline
        \textbf{服务管理} & 使用service命令 & 使用systemctl命令 \\
        \hline
        \textbf{运行级别} & 使用runlevel概念(0-6) & 使用target概念替代运行级别 \\
        \hline
        \textbf{配置文件} & 位于/etc/init.d/,使用脚本 & 位于/etc/systemd/system/,使用unit文件 \\
        \hline
        \textbf{依赖管理} & 手动管理服务依赖关系 & 自动管理服务依赖关系 \\
        \hline
        \textbf{开机速度} & 较慢,因为串行启动 & 较快,因为并行启动 \\
        \hline
        \textbf{日志管理} & 使用syslog & 集成journald日志系统 \\
        \hline
        \textbf{按需启动} & 不支持 & 支持服务的按需启动 \\
        \hline
        \textbf{系统状态} & 通过runlevel命令查看 & 通过systemctl status命令查看 \\
        \hline
        \textbf{默认采用} & RHEL 7及之前版本 & RHEL 8及之后版本 \\
        \hline
    \end{tabularx}
    \caption{systemd与System V init的区别}
    \label{tab:systemd_vs_sysvinit}
\end{table}

\subsection{systemctl常用命令}
\begin{table}[htbp]
    \centering
    \begin{tabularx}{\textwidth}{|X|X|X|}
        \hline
        \textbf{命令} & \textbf{功能描述} & \textbf{示例} \\
        \hline
        \textbf{systemctl status} & 查看服务状态 & systemctl status sshd \\
        \hline
        \textbf{systemctl start} & 启动服务 & systemctl start sshd \\
        \hline
        \textbf{systemctl stop} & 停止服务 & systemctl stop sshd \\
        \hline
        \textbf{systemctl restart} & 重启服务 & systemctl restart sshd \\
        \hline
        \textbf{systemctl reload} & 重新加载服务配置 & systemctl reload sshd \\
        \hline
        \textbf{systemctl enable} & 启用服务开机自启 & systemctl enable sshd \\
        \hline
        \textbf{systemctl disable} & 禁用服务开机自启 & systemctl disable sshd \\
        \hline
        \textbf{systemctl is-enabled} & 检查服务是否开机自启 & systemctl is-enabled sshd \\
        \hline
        \textbf{systemctl list-units} & 列出所有活动的单元 & systemctl list-units \\
        \hline
        \textbf{systemctl list-unit-files} & 列出所有单元文件 & systemctl list-unit-files \\
        \hline
        \textbf{systemctl list-dependencies} & 查看服务依赖关系 & systemctl list-dependencies sshd \\
        \hline
        \textbf{systemctl isolate} & 切换到指定目标 & systemctl isolate multi-user.target \\
        \hline
        \textbf{systemctl get-default} & 查看默认目标 & systemctl get-default \\
        \hline
        \textbf{systemctl set-default} & 设置默认目标 & systemctl set-default graphical.target \\
        \hline
        \textbf{systemctl poweroff} & 关机 & systemctl poweroff \\
        \hline
        \textbf{systemctl reboot} & 重启 & systemctl reboot \\
        \hline
        \textbf{systemctl suspend} & 挂起系统 & systemctl suspend \\
        \hline
        \textbf{systemctl hibernate} & 使系统休眠 & systemctl hibernate \\
        \hline
    \end{tabularx}
    \caption{systemctl常用命令}
    \label{tab:systemctl_commands}
\end{table}

\subsection{timedatectl命令}
timedatectl命令对RHEL /CentOS 7的分布式系统来说,是一个新工具,RHEL 8仍然沿用。

timedatectl命令作为systemd系统和服务管理器的一部分,代替旧的、传统的、用于基于Linux 分布式系统的sysvinit守护进程的date命令。

timedatectl命令用于管理系统时间和日期设置。timedatectl命令可以查询和更改系统时钟和设置,可以使用此命令来设置或更改当前的日期、时间和时区,或实现与远程NTP服务器的自动系统时钟同步。

它提供了以下功能:

\begin{itemize}
    \item 查询当前系统时间和日期配置
    \item 设置或更改系统时间和日期
    \item 管理系统时区设置
    \item 实现与远程NTP服务器的自动系统时钟同步
    \item 管理硬件时钟设置
\end{itemize}

\begin{verbatim}
# 查看当前系统时间和日期配置
timedatectl

# 查看所有可用的时区
timedatectl list-timezones

# 设置系统时区为上海
timedatectl set-timezone Asia/Shanghai

# 设置系统时间
timedatectl set-time "2024-01-27 14:30:00"

# 启用NTP时间同步
timedatectl set-ntp yes

# 禁用NTP时间同步
timedatectl set-ntp no

# 查看NTP同步状态
timedatectl status
\end{verbatim}

\textbf{注意事项}:
- timedatectl命令是systemd的一部分,在使用systemd的现代Linux发行版中可用
- 对于传统的System V init系统,需要使用date和hwclock等命令来管理时间
- 当启用NTP同步后,系统会自动从时间服务器获取准确时间

\subsection{clock命令}
clock命令用于从计算机的硬件获得日期和时间,它是hwclock命令的别名。在传统的System V init系统中,clock命令是管理硬件时钟的主要工具。

clock命令的功能:
\begin{itemize}
    \item 显示硬件时钟的当前时间
    \item 设置硬件时钟的时间
    \item 将系统时间同步到硬件时钟
    \item 将硬件时钟同步到系统时间
\end{itemize}

\begin{verbatim}
# 显示硬件时钟的当前时间
clock

# 显示硬件时钟的当前时间(使用详细格式)
clock --show

# 将系统时间同步到硬件时钟
clock --systohc

# 将硬件时钟同步到系统时间
clock --hctosys

# 设置硬件时钟的时间
clock --set --date="2024-01-27 14:30:00"
\end{verbatim}

clock命令的常用选项:
\begin{itemize}
    \item --show:显示硬件时钟的当前时间
    \item --set:设置硬件时钟的时间
    \item --date:指定日期和时间
    \item --systohc:将系统时间同步到硬件时钟
    \item --hctosys:将硬件时钟同步到系统时间
    \item --utc:使用UTC时间
    \item --localtime:使用本地时间
\end{itemize}

\textbf{使用说明}:clock命令和hwclock命令功能完全相同,它们是彼此的别名。在现代Linux发行版中,虽然systemd提供了timedatectl命令来管理时间,但clock/hwclock命令仍然可用,用于直接操作硬件时钟。

\subsection{重置root管理员密码}
重启Linux主机并出现引导界面时,按"e"键进入内核编辑界面。

在linux参数行的最后面追加"rd.break"参数,然后按"Ctrl + X"组合键来运行修改过的内核程序。

大约30s后,系统进入紧急救援模式。依次输入以下命令,等待系统重启操作完毕,就可以使用新密码newredhat来登录Linux操作系统了。

\begin{verbatim}
mount -o remount,rw /sysroot
chroot /sysroot
passwd
touch ./autorelabel
exit
reboot
\end{verbatim}

\section{多重引导配置}
Linux和Windows的多重引导(多系统引导)有多种实现方式,常用的有3种。

在这3种实现方式中,目前用户使用最多的是通过Linux的GRUB或者LILO实现Windows、Linux多重引导。

\chapter{Linux命令行基础}

\section{命令行界面简介}
Linux命令是对Linux操作系统进行管理的命令。对于Linux操作系统来说,无论是中央处理器、内存、磁盘驱动器、键盘、鼠标,还是用户等,都是文件。Linux命令是Linux 正常运行的核心,与dos命令类似。

\subsection{Linux命令的详细说明}
\begin{enumerate}
    \item "一切皆文件"哲学:Linux确实将几乎所有资源(包括硬件设备、进程、用户等)都抽象为文件,这是其设计的核心思想之一。这种设计使得系统管理更加统一和灵活。
    \item 命令的执行机制:Linux命令通常通过shell(如Bash)解释执行,shell作为命令解释器,将用户输入的命令转换为内核可理解的指令。
    \item 命令的分类:
        \begin{itemize}
            \item 内置命令:shell内置的命令(如cd、echo)
            \item 外部命令:独立的可执行文件(如ls、cp)
            \item 系统调用:直接与内核交互的底层命令
        \end{itemize}
    \item 与DOS命令的区别:虽然两者都是命令行界面,但Linux命令在功能、语法和灵活性方面更为强大,支持管道、重定向、通配符等高级特性。
    \item 命令的组成:完整的Linux命令通常包括命令名、选项和参数三部分,如ls -la /home。
    \item 命令的大小写敏感性:在Linux操作系统中,命令区分大小写。
    \item 命令自动补齐:在命令行中,可以使用“Tab”键来自动补齐命令,即可以只输入命令的前几个字母,然后按“Tab”键补齐。按“Tab”键时,如果系统只找到一个与输入字符相匹配的目录或文件,则自动补齐;如果没有匹配的内容或有多个相匹配的名字,系统将发出警鸣声,再按“Tab”键将列出所有相匹配的内容(如果有),以供用户选择。
    \item 历史命令:利用向上或向下的方向键,可以翻查曾经执行过的命令,并可以再次执行。
    \item 多命令执行:如果要在一个命令行上输入和执行多条命令,可以使用分号来分隔命令,如“cd /;ls”。
    \item 长命令换行:如果要断开一个长命令行,可以使用反斜杠“\textbackslash”。它可以将一个较长的命令分成多行表达,增强命令的可读性。执行后,shell自动显示提示符“>”,表示正在输入一个长命令,此时可继续在新的命令行上输入命令的后续部分。
    \item 后台执行命令:一个文本控制台或一个仿真终端在同一时刻只能执行一个程序或命令。在执行结束前,一般不能进行其他操作。此时可采用在后台执行程序的方式,以释放控制台或终端,使其仍能进行其他操作。要使程序以后台方式执行,只需在要执行的命令后跟上一个“\&”符号即可,如“top \&”。
\end{enumerate}

\begin{itemize}
    \item 终端与Shell的概念
    \item 常见Shell(Bash,Zsh,Fish等)
    \item 命令行提示符结构
    \item 快捷键与命令历史
\end{itemize}

\section{文件系统与目录结构}
\begin{itemize}
    \item Linux文件系统层次结构(FHS)
    \item 根目录(/)下主要子目录的功能
    \item 绝对路径与相对路径
    \item 文件与目录的基本概念
\end{itemize}

\section{常用文件与目录操作命令}
\begin{itemize}
    \item 浏览目录:ls,cd,pwd
    \item 创建与删除:mkdir,touch,rm,rmdir
    \item 复制与移动:cp,mv
    \item 查看文件内容:cat,more,less,head,tail
    \item 文件属性:chmod,chown,chgrp
    \item 文件搜索:find,grep
\end{itemize}

\subsection{pwd命令}
pwd(print working directory)命令用于显示当前工作目录的绝对路径。它是一个内置命令,无需参数即可执行。

\begin{verbatim}
# 显示当前工作目录的绝对路径
pwd

# 示例输出:
# /home/user/Documents
\end{verbatim}

pwd命令的主要作用:
\begin{itemize}
    \item 确认当前所在的目录位置
    \item 在执行复杂操作前验证工作目录
    \item 在脚本中获取当前目录路径
    \item 帮助用户了解目录结构
\end{itemize}

pwd命令没有常用的选项,直接执行即可显示当前工作目录的完整路径。

\subsection{clear命令}
clear命令用于清除终端屏幕上的内容,使终端回到干净的状态。它是一个内置命令,常用于在执行命令前清理屏幕,提高命令输出的可读性。

\begin{verbatim}
# 清除终端屏幕
clear

# 使用快捷键Ctrl+L也可以清除屏幕(与clear命令功能相同)
^L
\end{verbatim}

clear命令的功能:
\begin{itemize}
    \item 清除终端屏幕上的所有内容
    \item 保持当前的工作目录和shell环境不变
    \item 光标位置重置到屏幕左上角
\end{itemize}

\textbf{注意事项}:
\begin{itemize}
    \item clear命令只是清除屏幕显示,不会删除任何实际数据
    \item 清除后,使用向上方向键仍然可以查看之前的命令历史
    \item 不同终端模拟器对clear命令的处理可能略有差异
    \item 在某些终端中,clear命令可能只是将屏幕内容向上滚动,而不是真正清除
\end{itemize}

\subsection{cd命令}
cd(change directory)命令用来在不同的目录中进行切换。它是一个内置命令,用于改变当前工作目录。

\begin{verbatim}
# 切换到指定目录
cd /path/to/directory

# 切换到用户的家目录
cd
cd ~

# 切换到上一级目录
cd ..

# 切换到上一次的工作目录
cd -
\end{verbatim}

cd命令的特点:
\begin{itemize}
    \item 用户在登录系统后,会处于用户的"家目录"(\$HOME)中
    \item 家目录一般以/home开始,后接用户名(如/home/user)
    \item root用户的家目录为/root
    \item 可以使用绝对路径或相对路径指定目标目录
    \item 支持特殊符号:
        \begin{itemize}
            \item "." 代表当前目录
            \item ".." 代表当前目录的父目录
            \item "~" 代表用户的家目录(主目录)
            \item "-" 代表上一次的工作目录
        \end{itemize}
    \item 不带任何参数的"cd"命令相当于"cd ~",即将目录切换到用户的家目录
\end{itemize}

cd命令是Linux中最常用的命令之一,是用户在文件系统中导航的基础工具。

\subsection{ls命令}
ls(list)命令用来列出文件或目录信息。它是一个常用的文件系统导航工具,可以显示指定目录中的文件和子目录。

\begin{verbatim}
# 列出当前目录中的文件和子目录
ls

# 列出详细信息(包括权限、所有者、大小、修改时间等)
ls -l

# 显示隐藏文件(以.开头的文件)
ls -a

# 以人类可读的格式显示文件大小
ls -h

# 按修改时间排序(最新的在前)
ls -t

# 递归列出子目录中的内容
ls -R
\end{verbatim}

ls命令的常用选项:
\begin{itemize}
    \item -l:显示详细信息
    \item -a:显示所有文件(包括隐藏文件)
    \item -A:显示所有文件(包括隐藏文件,但不显示.和..)
    \item -h:以人类可读的格式显示文件大小
    \item -t:按修改时间排序
    \item -c:按状态改变时间排序
    \item -R:递归显示子目录内容
    \item -i:显示inode号
    \item -S:按文件大小排序
    \item -r:反向排序
    \item -F:在文件名后添加类型指示符(/表示目录,*表示可执行文件,@表示符号链接)
    \item -C:按列格式显示文件列表
    \item -d:只显示目录本身,不显示目录中的内容
    \item -g:类似-l,但不显示所有者信息
\end{itemize}

ls命令是Linux中最基本、最常用的命令之一,通过不同的选项组合,可以满足各种文件浏览需求。

\subsection{cat命令}
cat(concatenate)命令主要用于滚动显示文件内容,或将多个文件合并成一个文件。它是一个常用的文件内容查看工具。

通常使用cat命令查看文件内容,但是cat命令的输出内容不能分页显示,要查看超过一屏的文件内容,需要使用more或less等其他命令。如果在cat命令中没有指定参数,则cat会从标准输入(键盘)中获取内容。

\begin{verbatim}
# 显示单个文件的内容
cat file.txt

# 显示多个文件的内容
cat file1.txt file2.txt

# 将多个文件合并到一个新文件
cat file1.txt file2.txt > combined.txt

# 将内容追加到文件末尾
cat file.txt >> existing.txt

# 从标准输入读取内容并显示
cat
\end{verbatim}

cat命令的常用选项:
\begin{itemize}
    \item -n:显示行号
    \item -b:显示行号(但不包括空白行)
    \item -s:将连续的空白行压缩为一个
    \item -A:显示所有字符(包括控制字符)
\end{itemize}

cat命令是Linux中最基本的文件内容查看命令之一,适用于查看小型文件的完整内容。

\subsection{more命令}
more命令通常用于分屏显示文件内容。在使用cat命令时,如果文件内容太长,则用户只能看到文件的最后一部分。这时可以使用more命令一页一页地分屏显示文件内容。

\begin{verbatim}
# 查看文件内容(分屏显示)
more file.txt

# 从指定行开始查看
more +10 file.txt

# 分页显示文件内容,并指定每屏显示10行
more -10 file.txt
\end{verbatim}

more命令的操作键:
\begin{itemize}
    \item Enter键:向下移动一行
    \item Space键:向下移动一页
    \item Q键:退出more命令
    \item B键:向上移动一页
    \item /键:搜索关键字
    \item n键:查找下一个匹配项
\end{itemize}

在大部分情况下,可以不加任何选项直接执行more命令查看文件内容。执行more命令后,进入more状态,可以使用上述操作键进行导航。

more命令经常在管道中被调用,以实现各种命令输出内容的分屏显示。例如:

\begin{verbatim}
# 查看目录内容并分屏显示
ls -la | more

# 查看系统日志并分屏显示
tail -n 100 /var/log/syslog | more
\end{verbatim}

\subsection{less命令}
less命令是more命令的改进版,比more命令的功能强大。more命令只能向下翻页,而less命令不但可以向下、向上翻页,还可以前后左右移动。

\begin{verbatim}
# 查看文件内容(分屏显示,支持上下滚动)
less file.txt

# 查看文件内容并显示行号
less -N file.txt

# 在管道中使用less
echo "Hello\nWorld\nLinux" | less
\end{verbatim}

less命令的操作键:
\begin{itemize}
    \item Enter键:向下移动一行
    \item Space键:向下移动一页
    \item B键:向上移动一页
    \item 方向键:向前、后、左、右移动
    \item Q键:退出less命令
    \item /键:搜索关键字,输入要查找的单词或字符后按Enter键,less会高亮显示第一个匹配项
    \item n键:查找下一个匹配项
    \item N键:查找上一个匹配项
    \item G键:跳转到文件末尾
    \item g键:跳转到文件开头
\end{itemize}

less命令的搜索功能:
less命令还支持在一个文本文件中进行快速查找。先按"/"键,再输入要查找的单词或字符,按Enter键后,less命令会在文本文件中进行快速查找,并把找到的第一个搜索目标高亮显示。如果希望继续查找,就再次按"/"键,再按"Enter"键即可。

执行less命令后,进入less状态,可以使用上述操作键进行导航。less命令的功能比more命令更加丰富,是查看文件内容的常用工具。

\subsection{head命令}
head命令用于显示文件的开头部分,默认情况下只显示文件前10行的内容。

head命令的常用选项:
\begin{itemize}
    \item -n:指定显示的行数。若num为负值,则表示从倒数|num|行后面的所有行不显示
    \item -c:指定显示的字节数
\end{itemize}

\begin{verbatim}
# 查看文件的前10行(默认)
head file.txt

# 查看文件的前5行(使用-n选项)
head -5 file.txt
head -n 5 file.txt

# 查看文件的前20行
head -20 file.txt
head -n 20 file.txt

# 查看文件的前100个字节(使用-c选项)
head -c 100 file.txt

# 查看文件中除了最后3行以外的所有行(使用-n负值)
head -n -3 file.txt
\end{verbatim}

\subsection{tail命令}
tail命令用于显示文件内容的末尾部分,默认情况下,只显示文件内容的末尾10行。

tail命令的常用选项:
\begin{itemize}
    \item -n num:显示指定文件内容的末尾num行
    \item -c num:显示指定文件内容的末尾num个字符
    \item -n+num:从第num行开始显示指定文件的内容
    \item -f:持续刷新文件内容,实时显示文件的最新变化
\end{itemize}

tail命令"最强悍"的功能是可以持续刷新一个文件的内容,想要实时查看最新日志文件时,这个功能特别有用。使用-f选项可以实时监控文件的变化,当文件有新内容添加时,会自动显示出来。

\begin{verbatim}
# 查看文件的最后10行(默认)
tail file.txt

# 查看文件的最后5行(使用-n选项)
tail -5 file.txt
tail -n 5 file.txt

# 查看文件的最后20行
tail -20 file.txt
tail -n 20 file.txt

# 查看文件的最后100个字符(使用-c选项)
tail -c 100 file.txt

# 从第20行开始显示文件内容(使用-n+num格式)
tail -n+20 file.txt

# 持续刷新文件内容,实时显示最新变化(使用-f选项)
tail -f file.txt

# 实时查看系统日志
tail -f /var/log/syslog
\end{verbatim}

\subsection{mkdir命令}
mkdir命令用于创建一个目录。目录名可以为相对路径,也可以为绝对路径。

mkdir命令的常用选项:
\begin{itemize}
    \item -p:在创建目录时,如果父目录不存在,则同时创建该目录及该目录的父目录
\end{itemize}

\begin{verbatim}
# 在当前目录创建一个新目录
mkdir new_directory

# 创建多个目录
mkdir dir1 dir2 dir3

# 创建嵌套目录(使用-p选项)
mkdir -p parent/child/grandchild

# 使用绝对路径创建目录
mkdir /path/to/new_directory
\end{verbatim}

\subsection{rmdir命令}
rmdir命令用于删除空目录。目录名可以为相对路径,也可以为绝对路径。但所删除的目录必须为空目录。

rmdir命令的常用选项:
\begin{itemize}
    \item -p:在删除目录时,一同删除父目录,但父目录中必须没有其他目录及文件
\end{itemize}

\begin{verbatim}
# 删除一个空目录
rmdir empty_directory

# 删除多个空目录
rmdir dir1 dir2 dir3

# 删除目录及其空的父目录(使用-p选项)
rmdir -p parent/child/grandchild
\end{verbatim}

\subsection{rm命令}
rm命令用于删除文件或目录。默认情况下,rm命令只能删除文件,不能删除目录。

rm命令的常用选项:
\begin{itemize}
    \item -a:删除所有文件,包括隐藏文件
    \item -f:强制删除,不提示确认
    \item -i:删除前提示用户确认
    \item -R:递归删除目录及其内容
\end{itemize}

\begin{verbatim}
# 删除一个文件
rm file.txt

# 删除多个文件
rm file1.txt file2.txt file3.txt

# 删除目录及其所有内容(使用-R选项)
rm -R directory

# 强制删除文件(使用-f选项)
rm -f file.txt

# 删除前提示确认(使用-i选项)
rm -i file.txt

# 删除所有文件,包括隐藏文件(使用-a选项)
rm -a *
\end{verbatim}

\subsection{cp命令}
cp命令用于复制文件或目录。可以将一个或多个文件复制到指定的目标文件或目录中。这个命令是非常重要的,不同身份执行这个命令会有不同的结果产生,尤其是-a、-p选项,对于不同身份来说,差异非常大。

在预设的条件中,cp命令的源文件与目的文件的权限是不同的,目的文件的拥有者通常会是命令操作者本身。
想要复制文件给其他用户,也必须要注意文件的权限(包含读、写、执行以及文件拥有者等),否则,其他用户还是无法对你给的文件进行修改。

cp命令的常用选项:
\begin{itemize}
    \item -p:保留源文件的属性(权限、时间戳等)
    \item -r:递归复制目录及其内容
    \item -i:交互式复制,覆盖前提示确认
    \item -a:相当于-pdr,保留所有属性并递归复制
\end{itemize}

\begin{verbatim}
# 复制文件
cp source.txt destination.txt

# 复制文件到目录
cp file.txt directory/

# 复制多个文件到目录
cp file1.txt file2.txt file3.txt directory/

# 复制目录(使用-r选项)
cp -r source_directory destination_directory

# 保留属性复制(使用-p选项)
cp -p file.txt file_backup.txt

# 交互式复制(使用-i选项)
cp -i file.txt file.txt

# 保留所有属性并递归复制(使用-a选项)
cp -a source_dir target_dir
\end{verbatim}

\subsection{mv命令}
mv命令主要用于文件或目录的移动或改名。

mv命令的常用选项:
\begin{itemize}
    \item -i:交互式操作,覆盖前提示确认
    \item -f:强制覆盖,不提示确认
    \item -v:详细模式,显示移动过程
\end{itemize}

\begin{verbatim}
# 移动文件到另一个目录
mv file.txt directory/

# 文件改名
mv old_name.txt new_name.txt

# 移动目录
mv source_dir destination_dir

# 交互式移动(使用-i选项)
mv -i file.txt directory/

# 强制移动(使用-f选项)
mv -f file.txt directory/
\end{verbatim}

\subsection{touch命令}
touch命令用于建立文件或更新文件的修改日期。

touch命令的常用选项:
\begin{itemize}
    \item -a:只更新文件的访问时间
    \item -m:只更新文件的修改时间
    \item -d:指定文件的时间戳
    \item -t:使用[[CC]YY]MMDDhhmm[.ss]格式指定时间戳
\end{itemize}

\begin{verbatim}
# 创建新文件
 touch new_file.txt

# 更新文件的修改日期
touch existing_file.txt

# 同时更新多个文件的时间戳
touch file1.txt file2.txt file3.txt

# 只更新访问时间(使用-a选项)
touch -a file.txt

# 只更新修改时间(使用-m选项)
touch -m file.txt

# 指定时间戳(使用-d选项)
touch -d "2023-12-31 23:59:59" file.txt
\end{verbatim}

\subsection{whereis命令}
whereis命令用来寻找命令的可执行文件所在的位置。

whereis命令的常用选项:
\begin{itemize}
    \item -b:只查找可执行文件
    \item -m:只查找手册页
    \item -s:只查找源代码文件
    \item -u:查找不常见的条目
\end{itemize}

\begin{verbatim}
# 查找命令的可执行文件、手册页和源代码
whereis ls

# 只查找可执行文件(使用-b选项)
whereis -b ls

# 只查找手册页(使用-m选项)
whereis -m ls

# 查找多个命令的位置
whereis ls grep find
\end{verbatim}

\subsection{whatis命令}
whatis命令用于获取命令简介。它从某个程序的使用手册中抽出一行简单的介绍性文件,帮助用户迅速了解这个程序的具体功能。

\begin{verbatim}
# 获取命令简介
whatis ls

# 获取多个命令的简介
whatis ls grep find

# 获取系统调用的简介
whatis open read write
\end{verbatim}

\subsection{find命令}
find命令用于查找文件。它的功能非常强大。

find命令的常用选项:
\begin{itemize}
    \item -name:按文件名查找
    \item -type:按文件类型查找(f:普通文件,d:目录,l:符号链接,b:块设备,c:字符设备,p:命名管道,s:套接字)
    \item -size:按文件大小查找,-size n表示查找大小为n块的文件(一块为512B),+n表示大于n块,-n表示小于n块,nc表示n个字符
    \item -mtime:按修改时间查找
    \item -ctime:按更改时间查找
    \item -user:按文件所有者查找
    \item -group:按文件所属组查找
    \item -inum:按inode号查找
    \item -perm:按文件权限查找
    \item -newer:查找比指定文件更新的文件
    \item -exec:对找到的文件执行命令,{}表示找到的文件,\\;表示命令结束
    \item -ok:对找到的文件执行命令,执行前会询问确认
    \item -print:打印找到的文件路径(默认选项)
\end{itemize}

\begin{verbatim}
# 在当前目录查找名为test.txt的文件
find . -name "test.txt"

# 查找所有普通文件
find . -type f

# 查找大于1MB的文件
find . -size +1M

# 查找最近7天修改的文件
find . -mtime -7

# 查找属于root用户的文件
find . -user root

# 查找具有执行权限的文件
find . -perm /u+x

# 查找文件并执行命令(删除找到的文件)
find . -name "*.tmp" -exec rm {} \;

# 按文件所属组查找
find . -group users

# 显式使用-print选项(默认行为)
find . -name "*.txt" -print

# 按inode号查找文件
find . -inum 12345

# 按更改时间查找(最近7天更改的文件)
find . -ctime -7

# 按文件权限查找(精确匹配644权限)
find . -perm 644

# 查找比指定文件更新的文件
find . -newer reference.txt

# 使用-exec执行命令(删除找到的文件,无确认)
find . -name "*.tmp" -exec rm {} \;

# 使用-ok执行命令(删除找到的文件,执行前确认)
find . -name "*.tmp" -ok rm {} \;

# 后台运行find命令(添加&符号)
find . -name "*.log" -size +1M -exec gzip {} \; &
\end{verbatim}

\textbf{注意事项}:由于find命令在执行过程中将消耗大量资源,所以建议以后台方式运行(在命令末尾添加\&符号)。

\subsection{grep命令}
grep命令用于查找文件中包含指定字符串的行。

grep命令的常用选项:
\begin{itemize}
    \item -i:忽略大小写
    \item -n:显示行号
    \item -c:统计匹配行的数量
    \item -l:只显示包含匹配的文件名
    \item -h:不显示文件名(当搜索多个文件时)
    \item -v:反向查找,显示不包含指定字符串的行
    \item -r:递归查找子目录
    \item -A:显示匹配行及其后n行
    \item -B:显示匹配行及其前n行
    \item -C:显示匹配行及其前后n行
    \item -E:使用扩展正则表达式
\end{itemize}

\begin{verbatim}
# 在文件中查找指定字符串
grep "pattern" file.txt

# 忽略大小写查找
grep -i "pattern" file.txt

# 显示行号
grep -n "pattern" file.txt

# 反向查找
grep -v "pattern" file.txt

# 递归查找子目录
grep -r "pattern" directory/

# 显示匹配行及其前后2行
grep -C 2 "pattern" file.txt

# 统计匹配行的数量
grep -c "pattern" file.txt

# 只显示包含匹配的文件名
grep -l "pattern" *.txt

# 不显示文件名(搜索多个文件时)
grep -h "pattern" *.txt

# 使用正则表达式符号(^表示行开始,$表示行结尾)
grep "^pattern" file.txt  # 查找以pattern开头的行
grep "pattern$" file.txt  # 查找以pattern结尾的行

# 查找包含空格的字符串
grep "hello world" file.txt  # 使用双引号
grep 'hello world' file.txt  # 使用单引号

# 过滤掉带"#"的注释行和空白行
grep -v "^#" file.txt | grep -v "^$"
\end{verbatim}

\textbf{使用说明}:在grep命令中,字符"\^"表示行的开始,字符"\$"表示行的结尾。如果要查找的字符串中带有空格,则可以用单引号或双引号标注。

\textbf{命令比较}:grep命令和find命令的差别在于,grep命令是在文件中搜索满足条件的行,而find命令是在指定目录下根据文件的相关信息查找满足指定条件的文件。

\subsection{dd命令}
dd命令是一个比较重要而且有特色的命令,它能够让用户按照指定大小和数量的数据块来复制文件的内容。当然如果需要,还可以在复制过程中转换其中的数据。

dd命令的功能也绝不仅限于复制文件这么简单。如果想把光驱设备中的光盘制作成iso映像文件,在Windows操作系统中需要借助于第三方软件才能做到,但在Linux操作系统中可以直接使用dd命令来压制映像文件,将它变成一个可立即使用的iso映像文件。

Linux操作系统中有一个名为/dev/zero的设备文件,因为这个文件不会占用系统存储空间,却可以提供无穷无尽的数据,所以可以使用它作为dd命令的输入文件来生成一个指定大小的文件。

dd命令的常用选项:
\begin{itemize}
    \item if:输入文件(默认为标准输入)
    \item of:输出文件(默认为标准输出)
    \item bs:数据块大小(默认512字节)
    \item count:复制的数据块数量
    \item seek:输出文件开始时跳过的块数
    \item skip:输入文件开始时跳过的块数
    \item conv:转换方式(如notrunc, noerror, sync等)
    \item status:显示进度信息(如status=progress)
\end{itemize}

\begin{verbatim}
# 基本复制文件
cp file1.txt file2.txt

# 使用dd复制文件(更精确控制)
dd if=file1.txt of=file2.txt bs=1M count=1

# 创建指定大小的空文件(1GB)
dd if=/dev/zero of=emptyfile.img bs=1M count=1024

# 将光驱制作成iso映像文件
dd if=/dev/cdrom of=disc_image.iso bs=2048 status=progress

# 备份MBR(主引导记录)
dd if=/dev/sda of=mbr_backup.img bs=512 count=1

# 克隆磁盘分区
dd if=/dev/sda1 of=/dev/sdb1 bs=4M status=progress

# 测试磁盘读写速度
time dd if=/dev/zero of=testfile bs=1M count=1000 oflag=direct
time dd if=testfile of=/dev/null bs=1M count=1000 iflag=direct
rm testfile
\end{verbatim}

\textbf{注意事项}:
- dd命令的参数顺序不影响执行结果,但建议使用"if=输入文件 of=输出文件"的格式以提高可读性
- 执行dd命令时要特别小心,尤其是操作磁盘设备时,错误的参数可能导致数据丢失
- 使用status=progress选项可以在执行过程中显示复制进度

\subsection{cal命令}
cal命令用于显示指定月份或年份的日历。它是一个简单但实用的工具,可以帮助用户快速查看日期信息。

cal命令的使用规则:
\begin{itemize}
    \item 不带任何参数:显示当前月份的日历
    \item 一个参数:表示年份(范围为1~9999),显示该年份的完整日历
    \item 两个参数:第一个表示月份,第二个表示年份,显示指定月份的日历
\end{itemize}

\begin{verbatim}
# 显示当前月份的日历
cal

# 显示2024年的完整日历
cal 2024

# 显示2024年12月的日历
cal 12 2024
\end{verbatim}

cal命令的常用选项:
\begin{itemize}
    \item -1:只显示当前月份(默认)
    \item -3:显示上个月、当前月和下个月的日历
    \item -y:显示当前年份的完整日历
    \item -j:使用儒略日(一年中的第几天)显示日历
    \item -m:以周一为一周的开始
\end{itemize}

\begin{verbatim}
# 显示上个月、当前月和下个月的日历
cal -3

# 显示当前年份的完整日历
cal -y

# 使用儒略日显示当前月份的日历
cal -j

# 以周一为一周的开始显示日历
cal -m
\end{verbatim}

\textbf{使用说明}:cal命令的输出会自动调整格式,使其在终端中显示美观。当指定年份时,会显示该年份的所有月份;当指定月份和年份时,会只显示该月份的日历。


\section{文本编辑}
\begin{itemize}
    \item Vi/Vim编辑器基础
    \item Nano编辑器使用
    \item 常用编辑命令与快捷键
\end{itemize}

\chapter{Linux用户与权限管理}

\section{用户与组的概念}
\begin{itemize}
    \item 用户(User)与用户ID(UID)
    \item 组(Group)与组ID(GID)
    \item 主要组与附加组
    \item /etc/passwd,/etc/shadow,/etc/group文件解析
\end{itemize}

\section{用户与组管理命令}
\begin{itemize}
    \item 用户管理:useradd,usermod,userdel,passwd
    \item 组管理:groupadd,groupmod,groupdel
    \item 权限查看与修改:id,su,sudo,chmod,chown,chgrp
\end{itemize}

\section{文件权限机制}
\begin{itemize}
    \item 权限表示方法(rwx,数字表示法)
    \item 文件权限与目录权限的区别
    \item 特殊权限(SUID,SGID,Sticky Bit)
    \item ACL(访问控制列表)
\end{itemize}

\chapter{Linux软件包管理}

\section{软件包管理概述}
\begin{itemize}
    \item 软件包的概念
    \item 依赖关系管理
    \item 主流软件包管理系统
\end{itemize}

\section{RPM包管理系统}
\begin{itemize}
    \item RPM命令基本用法(rpm)
    \item YUM包管理器(CentOS/RHEL)
    \item DNF包管理器(新一代YUM)
\end{itemize}

\subsection{rpm命令}
rpm命令主要用于对RPM软件包进行管理。RPM软件包是Linux的各种发行版中应用最为广泛的软件包格式之一。

rpm命令的常用选项:
\begin{itemize}
    \item -i:安装RPM软件包
    \item -e:卸载RPM软件包
    \item --nodeps:卸载时不检查依赖性
    \item -U:升级RPM软件包
    \item -F:freshen模式,只升级已安装的软件包
    \item -v:详细模式,显示详细的执行过程
    \item -h:显示安装进度条
    \item -q:查询RPM软件包
    \item -a:查询所有已安装的软件包
    \item -l:列出软件包中的文件
    \item -p:对未安装的软件包进行操作
    \item -i:与-q配合使用,显示软件包的详细信息(-qi)
    \item -l:与-q配合使用,列出软件包中的文件(-ql)
    \item -f:与-q配合使用,查询文件属于哪个软件包(-qf)
    \item -p:与-q配合使用,查询未安装软件包的信息(-qp)
\end{itemize}

\begin{verbatim}
# 安装RPM软件包
rpm -i package.rpm

# 详细模式安装(使用-v选项)
rpm -iv package.rpm

# 显示进度条安装(使用-vh选项)
rpm -ivh package.rpm

# 升级RPM软件包
rpm -U package.rpm

# 详细模式升级(使用-vh选项)
rpm -Uvh package.rpm

# Freshen模式升级(只升级已安装的包,使用-F选项)
rpm -Fvh package.rpm

# 卸载RPM软件包(注意:卸载时不需要加.rpm扩展名)
rpm -e package_name

# 详细模式卸载(使用-v选项)
rpm -ev package_name

# 不检查依赖性卸载(使用--nodeps选项)
rpm -e --nodeps package_name

# 示例:卸载network-scripts软件包且不检查依赖性
# 注意:软件包名称会因系统版本而稍有差异,不要机械照抄
rpm -e network-scripts-10.00.6-1.el8.x86_64 --nodeps

# 查询已安装的软件包
rpm -q package_name

# 查询所有已安装的软件包
rpm -qa

# 列出软件包中的文件
rpm -ql package_name

# 查询未安装软件包的信息
rpm -qp package.rpm

# 显示软件包的详细信息(使用-qi选项)
rpm -qi package_name

# 列出软件包中的文件(使用-ql选项)
rpm -ql package_name

# 查询文件属于哪个软件包(使用-qf选项)
rpm -qf /path/to/file

# 查询未安装软件包的详细信息(使用-qpi选项)
rpm -qpi package.rpm
\end{verbatim}

尽管RPM命令能够帮助用户查询软件相关的依赖关系,但具体问题还是要运维人员自己来解决。而有些大型软件可能与数十个程序都有依赖关系,在这种情况下安装软件是非常痛苦的。yum软件仓库便是为了进一步降低软件安装难度和复杂度而设计的软件。

\subsection{yum软件仓库}

RHEL先将发布的软件存放到yum服务器内,再分析这些软件的依赖属性问题,将软件内的记录信息写下来,然后将这些信息分析后记录成软件相关的清单列表。这些列表数据与软件所在的位置可以称为容器(repository)。当Linux客户端有软件安装的需求时,Linux客户端主机会主动向网络上的yum服务器的容器网址请求下载清单列表,然后通过清单列表的数据与本机RPM数据库已存在的软件数据相比较,就能够一次性安装所有需要的具有依赖属性的软件了。

当Linux客户端有升级、安装的需求时,会向容器要求更新清单列表,使清单列表更新到本机的/var/cache/yum中。当Linux客户端实施更新、安装时,会用清单列表的数据与本机的RPM数据库进行比较,这样就知道该下载什么软件了。接下来会到yum服务器下载所需要的软件,然后通过RPM的机制开始安装软件。这就是整个流程,仍然离不开RPM。

RHEL 8提供了基于Fedora 28中DNF的包管理系统yum v4,兼容RHEL 7的yum v3。

\subsection{常见DNF命令列表}

\begin{table}[htbp]
    \centering
    \begin{tabularx}{\textwidth}{|X|l|}
        \hline
        \textbf{功能描述} & \textbf{命令} \\
        \hline
        % 基本安装操作
        安装软件 & \texttt{dnf install 软件包名}(示例:\texttt{dnf install firefox}) \\
        自动安装软件(无需确认) & \texttt{dnf install -y 软件包名} \\
        卸载软件 & \texttt{dnf remove 软件包名}(示例:\texttt{dnf remove firefox}) \\
        自动卸载软件(无需确认) & \texttt{dnf remove -y 软件包名} \\
        重新安装软件 & \texttt{dnf reinstall 软件包名}(示例:\texttt{dnf reinstall firefox}) \\
        \hline
        % 更新操作
        更新软件 & \texttt{dnf update 软件包名} \\
        更新所有软件 & \texttt{dnf update} \\
        仅更新安全补丁 & \texttt{dnf update --security} \\
        检查更新 & \texttt{dnf check-update} \\
        \hline
        % 软件查询
        搜索软件 & \texttt{dnf search 关键词}(示例:\texttt{dnf search web browser}) \\
        查看软件信息 & \texttt{dnf info 软件包名}(示例:\texttt{dnf info firefox}) \\
        列出已安装软件 & \texttt{dnf list installed} \\
        列出可用软件 & \texttt{dnf list available} \\
        列出所有软件包(包括已安装和可用的) & \texttt{dnf list all} \\
        查看软件依赖 & \texttt{dnf deplist 软件包名} \\
        \hline
        % 系统维护
        清理缓存 & \texttt{dnf clean all} \\
        重新生成缓存 & \texttt{dnf makecache} \\
        \hline
        % 软件组管理
        查看软件组 & \texttt{dnf grouplist} \\
        查看软件组详细信息 & \texttt{dnf groupinfo 软件包组} \\
        安装软件组 & \texttt{dnf groupinstall 软件组名} \\
        卸载软件组 & \texttt{dnf groupremove 软件组名} \\
        \hline
        % 仓库管理
        查看仓库 & \texttt{dnf repolist} \\
        查看所有仓库(包括禁用的) & \texttt{dnf repolist all} \\
        启用仓库 & \texttt{dnf config-manager --set-enabled 仓库名} \\
        禁用仓库 & \texttt{dnf config-manager --set-disabled 仓库名} \\
        \hline
    \end{tabularx}
    \caption{常见DNF命令列表}
    \label{tab:dnf_commands}
\end{table}

\section{DPKG包管理系统}
\begin{itemize}
    \item DPKG命令基本用法(dpkg)
    \item APT包管理器(Debian/Ubuntu)
    \item 软件源配置
\end{itemize}

\section{源码编译安装}
\begin{itemize}
    \item 编译环境准备
    \item 源码获取与解压
    \item configure,make,make install流程
    \item 软件卸载
\end{itemize}

\chapter{Linux系统管理}

\section{进程管理}
\begin{itemize}
    \item 进程的概念与状态
    \item 进程管理命令:ps,top,htop,pstree
    \item 进程控制:kill,killall,pkill
    \item 作业管理:bg,fg,jobs,nohup
    \item 系统负载查看:uptime,w
\end{itemize}

\subsection{ps命令}
ps命令主要用于查看系统的进程。它是Process Status的缩写,是Linux系统中最常用的进程查看工具之一。

ps命令的功能:
\begin{itemize}
    \item 查看当前系统中运行的进程
    \item 查看进程的详细信息,如PID、PPID、CPU使用率、内存使用率等
    \item 查看指定用户的进程
    \item 查看指定终端的进程
    \item 查看进程的层级关系
\end{itemize}

ps命令的常用选项:
\begin{itemize}
    \item -a:显示所有用户的进程
    \item -u:显示进程的详细信息(包括用户、CPU、内存等)
    \item -x:显示没有控制终端的进程
    \item -e:显示所有进程,等同于-A
    \item -f:显示完整格式的进程信息
    \item -l:显示长格式的进程信息
    \item -o:自定义输出格式
    \item --forest:以树状结构显示进程间的关系
    \item -p:显示指定PID的进程
    \item -t:显示指定终端的进程,后面可跟终端设备名
    \item -u:显示指定用户的进程
    \item -w:设置输出宽度,允许更宽的输出显示,可多次使用以增加宽度
\end{itemize}

\begin{verbatim}
# 显示当前终端的进程
ps

# 显示所有进程的详细信息
ps aux

# 显示所有进程的完整格式信息
ps -ef

# 以树状结构显示进程间的关系
ps aux --forest

# 显示指定PID的进程
ps -p 1234

# 显示指定用户的进程
ps -u username

# 显示没有控制终端的进程
ps ax

# 自定义输出格式(PID、用户、命令)
ps -eo pid,user,cmd

# 按CPU使用率排序显示进程
ps aux --sort=-%cpu

# 按内存使用率排序显示进程
ps aux --sort=-%mem

# 显示指定终端的进程并使用宽输出格式
ps -w -t pts/0

# 显示所有终端的进程并使用更宽的输出格式(使用多个-w选项)
ps -ww -t
\end{verbatim}

ps命令输出字段说明:
\begin{itemize}
    \item PID:进程ID
    \item PPID:父进程ID
    \item USER:进程所有者
    \item \%CPU:CPU使用率
    \item \%MEM:内存使用率
    \item VSZ:虚拟内存大小
    \item RSS:常驻内存大小
    \item TTY:控制终端
    \item STAT:进程状态(R:运行,S:睡眠,D:不可中断睡眠,Z:僵尸进程,T:停止)
    \item START:进程启动时间
    \item TIME:进程占用CPU的时间
    \item COMMAND:进程命令
\end{itemize}

\textbf{使用说明}:ps命令是一个静态查看进程的工具,它只显示命令执行时的进程状态。如果需要实时查看进程状态,可以使用top命令。

\textbf{参数风格说明}:ps命令支持两种参数风格,这是由Unix的历史发展造成的:
\begin{itemize}
    \item \textbf{BSD风格}:参数前面不加 `-`,例如 `ps aux`
    \item \textbf{System V风格}:参数前面加 `-`,例如 `ps -ef`
\end{itemize}

\textbf{历史原因}:
- BSD风格源自Berkeley Software Distribution Unix的传统
- System V风格源自AT\&T System V Unix的传统

ps命令为了兼容这两种Unix分支的用户习惯,同时支持两种参数形式。在实际使用中,两种风格的参数可以混合使用,但为了可读性和一致性,建议选择一种风格并坚持使用。

\textbf{与管道和重定向配合使用}:ps命令通常和重定向、管道等命令一起使用,用于查找出所需的进程。

\begin{verbatim}
# 使用管道过滤包含特定关键字的进程
ps aux | grep ssh

# 使用管道和grep查找特定用户的进程
ps aux | grep username

# 使用管道和awk提取特定字段
ps aux | awk '{print $1, $2, $11}'

# 使用管道和sort按CPU使用率排序
ps aux | sort -nrk 3,3 | head -10

# 使用管道和sort按内存使用率排序
ps aux | sort -nrk 4,4 | head -10

# 将ps命令的输出重定向到文件
ps aux > process_list.txt

# 使用管道和wc统计进程数量
ps aux | wc -l

# 查找特定命令的进程
ps aux | grep "nginx"

# 查找并排除grep自身的进程
ps aux | grep nginx | grep -v grep
\end{verbatim}

这些组合使用方式可以帮助用户更高效地筛选和分析系统中的进程信息,特别是在系统进程数量较多的情况下。

\subsection{top命令}
top命令是一个实时的进程监控工具,它可以动态显示系统中运行的进程及其资源占用情况。与ps命令不同,top命令会持续更新显示内容,默认每5秒刷新一次。

top命令的功能:
\begin{itemize}
    \item 实时监控系统中运行的进程
    \item 显示进程的CPU使用率、内存使用率等资源占用情况
    \item 支持按不同字段排序进程
    \item 支持交互式操作
    \item 可以调整刷新间隔
\end{itemize}

\begin{verbatim}
# 启动top命令,默认每5秒刷新一次
top

# 设置刷新间隔为20秒
top -d 20

# 以批处理模式运行,输出一次后退出
top -b -n 1

# 只显示指定用户的进程
top -u username

# 只显示指定PID的进程
top -p 1234
\end{verbatim}

top命令的常用选项:
\begin{itemize}
    \item -d:指定刷新间隔,单位为秒
    \item -b:批处理模式,适合重定向到文件
    \item -n:指定刷新次数,完成后退出
    \item -u:只显示指定用户的进程
    \item -p:只显示指定PID的进程
    \item -H:显示线程信息
    \item -c:显示完整的命令行
\end{itemize}

\textbf{使用说明}:和ps命令不同,top命令可以实时监控进程的状况。top命令界面自动每5s刷新一次,也可以用"top -d 20",使得top命令界面每20s刷新一次。

在top命令界面中,可以使用以下按键进行交互式操作:
\begin{itemize}
    \item P:按CPU使用率排序
    \item M:按内存使用率排序
    \item N:按PID排序
    \item T:按CPU累计时间排序
    \item k:终止指定进程
    \item q:退出top命令
\end{itemize}

\subsection{htop命令}
htop是一个交互式的进程查看器,是top命令的增强版,提供了更友好的界面和更多功能。

htop命令的功能:
\begin{itemize}
    \item 彩色界面:使用不同颜色显示不同类型的进程和资源使用情况
    \item 交互式操作:支持鼠标点击和键盘导航
    \item 实时监控:动态显示系统资源使用情况
    \item 进程管理:可以直接在界面中终止、调整优先级等操作
    \item 树形视图:可以以树状结构显示进程间的关系
    \item 多列显示:可以同时显示多个CPU核心的使用情况
\end{itemize}

\begin{verbatim}
# 启动htop
htop

# 以树形视图启动
htop -t

# 只显示指定用户的进程
htop -u username

# 只显示指定PID的进程
htop -p 1234
\end{verbatim}

htop命令的常用选项:
\begin{itemize}
    \item -t:以树形视图显示进程
    \item -u:只显示指定用户的进程
    \item -p:只显示指定PID的进程
    \item -s:指定排序字段
    \item -d:指定刷新间隔(单位为秒)
\end{itemize}

\textbf{安装方法}:在大多数Linux发行版中,可以通过包管理器安装:
\begin{verbatim}
# Ubuntu/Debian
sudo apt install htop

# CentOS/RHEL
sudo yum install htop

# Fedora
sudo dnf install htop

# Arch Linux
sudo pacman -S htop
\end{verbatim}

\textbf{常用快捷键}:
\begin{itemize}
    \item F1:显示帮助信息
    \item F2:进入设置界面
    \item F3:搜索进程
    \item F4:过滤器设置
    \item F5:切换树形视图
    \item F6:选择排序字段
    \item F7:降低进程优先级(增加nice值)
    \item F8:提高进程优先级(减少nice值)
    \item F9:终止进程
    \item F10:退出htop
    \item 空格:标记/取消标记进程
    \item u:显示指定用户的进程
    \item P:按CPU使用率排序
    \item M:按内存使用率排序
    \item T:按时间排序
\end{itemize}

\textbf{与top命令的对比}:
\begin{itemize}
    \item 界面:htop提供彩色界面,top默认是单色
    \item 交互性:htop支持鼠标操作,top主要依赖键盘
    \item 功能:htop提供更多内置功能,如直接调整优先级
    \item 显示:htop默认显示所有CPU核心的使用情况
    \item 启动速度:top启动速度通常比htop快
\end{itemize}

\subsection{pidof命令}
pidof命令用于查询某个指定服务进程的进程号码值(Process Identifier,PID)。它可以快速查找指定进程名的所有运行实例的PID。

pidof命令的功能:
\begin{itemize}
    \item 查找指定进程名的所有运行实例的PID
    \item 支持同时查找多个进程名
    \item 可以与kill命令配合使用,快速终止指定进程
\end{itemize}

\begin{verbatim}
# 查找sshd进程的PID
pidof sshd

# 查找nginx进程的PID
pidof nginx

# 同时查找多个进程的PID
pidof sshd nginx

# 与kill命令配合使用,终止指定进程
kill $(pidof sshd)

# 强制终止指定进程
kill -9 $(pidof nginx)
\end{verbatim}

pidof命令的常用选项:
\begin{itemize}
    \item -s:只返回一个PID
    \item -c:只返回当前运行在相同root目录下的进程
    \item -x:同时查找shell脚本的进程
    \item -o:排除指定的PID
\end{itemize}

\begin{verbatim}
# 只返回一个PID
pidof -s sshd

# 只返回当前运行在相同root目录下的进程
pidof -c nginx

# 同时查找shell脚本的进程
pidof -x script.sh

# 排除指定的PID
pidof -o 1234 sshd
\end{verbatim}

\textbf{使用说明}:pidof命令是一个简单但实用的工具,它可以快速获取指定进程的PID,特别适合与kill命令配合使用来终止进程。与ps命令相比,pidof命令更加专注于获取PID,输出更加简洁。

\subsection{kill命令}
kill命令用于向指定进程发送信号,通常用于终止进程。它是Linux系统中最常用的进程控制命令之一。

kill命令的功能:
\begin{itemize}
    \item 向指定PID的进程发送信号
    \item 支持发送不同类型的信号
    \item 可以与其他命令配合使用,如pidof、ps等
\end{itemize}

\textbf{常用信号}:
\begin{itemize}
    \item 1 (SIGHUP):重新加载进程配置
    \item 2 (SIGINT):中断进程(相当于Ctrl+C)
    \item 9 (SIGKILL):强制终止进程,进程无法捕获此信号
    \item 15 (SIGTERM):终止进程(默认信号),进程可以捕获此信号并进行清理
    \item 18 (SIGCONT):继续暂停的进程
    \item 19 (SIGSTOP):暂停进程
\end{itemize}

\begin{verbatim}
# 使用默认信号(SIGTERM)终止进程
kill 1234

# 强制终止进程(使用SIGKILL信号)
kill -9 1234
kill -KILL 1234

# 重新加载进程配置(使用SIGHUP信号)
kill -1 1234
kill -HUP 1234

# 中断进程(使用SIGINT信号)
kill -2 1234
kill -INT 1234

# 暂停进程(使用SIGSTOP信号)
kill -19 1234
kill -STOP 1234

# 继续暂停的进程(使用SIGCONT信号)
kill -18 1234
kill -CONT 1234

# 与pidof命令配合使用,终止指定进程
kill $(pidof sshd)

# 与ps和grep配合使用,终止匹配的进程
kill $(ps aux | grep "nginx" | grep -v grep | awk '{print $2}')
\end{verbatim}

kill命令的常用选项:
\begin{itemize}
    \item -l:列出所有可用的信号
    \item -s:指定要发送的信号
    \item -n:指定信号的数字编号
\end{itemize}

\begin{verbatim}
# 列出所有可用的信号
kill -l

# 使用-s选项指定信号
kill -s TERM 1234

# 使用-n选项指定信号编号
kill -n 9 1234
\end{verbatim}

\textbf{注意事项}:
\begin{itemize}
    \item 默认情况下,kill命令发送SIGTERM信号(编号15),允许进程进行清理操作
    \item 使用SIGKILL信号(编号9)可以强制终止进程,但进程无法进行清理操作
    \item 只有进程的所有者或root用户才能向进程发送信号
    \item 某些系统进程可能受到保护,即使是root用户也无法终止
    \item 对于僵尸进程(Z状态),kill命令通常无效,因为这些进程已经终止但尚未被父进程回收
\end{itemize}

\subsection{killall命令}
killall命令用于通过进程名终止所有匹配的进程。与kill命令不同,killall命令不需要指定进程的PID,而是直接使用进程名。

killall命令的功能:
\begin{itemize}
    \item 通过进程名终止所有匹配的进程
    \item 支持发送不同类型的信号
    \item 可以指定用户,只终止特定用户的进程
    \item 支持交互式操作,确认后再终止进程
\end{itemize}

\begin{verbatim}
# 终止所有名为sshd的进程
killall sshd

# 强制终止所有名为nginx的进程
killall -9 nginx
killall -KILL nginx

# 交互式终止所有名为apache2的进程
killall -i apache2

# 只终止特定用户的进程
killall -u username sshd

# 向所有名为java的进程发送SIGHUP信号(重新加载配置)
killall -HUP java

# 终止所有与指定模式匹配的进程(使用正则表达式)
killall -r "^java.*"
\end{verbatim}

killall命令的常用选项:
\begin{itemize}
    \item -i:交互式操作,在终止进程前确认
    \item -9:强制终止进程,使用SIGKILL信号
    \item -u:只终止指定用户的进程
    \item -r:使用正则表达式匹配进程名
    \item -s:指定要发送的信号
    \item -v:详细模式,显示终止的进程
    \item -w:等待所有被终止的进程真正终止
\end{itemize}

\begin{verbatim}
# 详细模式终止进程
killall -v nginx

# 等待进程终止
killall -w sshd

# 使用指定信号
killall -s TERM apache2
\end{verbatim}

\textbf{使用场景}:通常来讲,复杂软件的服务程序会有多个进程协同为用户提供服务,如果逐个结束这些进程会比较麻烦,此时可以使用killall命令来批量结束某个服务程序带有的全部进程。

\textbf{注意事项}:
\begin{itemize}
    \item killall命令默认区分大小写,进程名必须完全匹配
    \item 只有进程的所有者或root用户才能终止进程
    \item 使用killall命令时要特别小心,避免误终止重要进程
    \item 在某些系统上,killall命令可能会终止与进程名部分匹配的进程,因此建议使用精确的进程名
\end{itemize}

\subsection{nice命令}
nice命令用于调整进程的优先级,通过设置进程的nice值来影响CPU分配。nice值的范围通常为-20到19,其中-20表示最高优先级,19表示最低优先级。默认情况下,新进程的nice值为0。

Linux操作系统有两个和进程有关的优先级:
\begin{itemize}
    \item PRI值:进程实际的优先级,它是由操作系统动态计算的。
    \item NI值:进程的nice值,由用户通过nice命令设置。
\end{itemize}

NI值的特性:
\begin{itemize}
    \item NI值可以被用户更改
    \item NI值越大,优先级越低
    \item 一般用户只能增大NI值(降低优先级),只有超级用户才可以减小NI值(提高优先级)
    \item NI值被改变后,会影响PRI值
    \item 优先级高的进程被优先运行
    \item 默认时进程的NI值为0
\end{itemize}

PRI值的计算与NI值有关。使用"ps -l"命令可以查看这两个优先级值。

nice命令的功能:
\begin{itemize}
    \item 调整新启动进程的优先级
    \item 通过设置nice值来影响CPU时间分配
    \item 普通用户只能降低进程优先级(增加nice值)
    \item root用户可以任意调整进程优先级
\end{itemize}

\begin{verbatim}
# 以默认优先级启动进程
nice -n 0 ./myprogram

# 降低进程优先级(增加nice值)
nice -n 10 ./myprogram

# 以更高优先级启动进程(需要root权限)
sudo nice -n -5 ./myprogram

# 使用简写形式(-10 表示 nice 值为 10)
nice -10 ./myprogram
\end{verbatim}

nice命令的常用选项:
\begin{itemize}
    \item -n:指定nice值,范围为-20到19
    \item --adjustment=N:与-n选项相同,指定nice值的调整量
\end{itemize}

\begin{verbatim}
# 使用--adjustment选项
nice --adjustment=5 ./myprogram
\end{verbatim}

\textbf{注意事项}:
\begin{itemize}
    \item 普通用户只能将进程的nice值增加(降低优先级),不能减少(提高优先级)
    \item root用户可以任意调整进程的nice值
    \item nice命令只能在启动进程时设置优先级,不能调整已运行进程的优先级
    \item 要调整已运行进程的优先级,需要使用renice命令
\end{itemize}

\subsection{renice命令}
renice命令是根据进程的进程号来改变进程优先级的。与nice命令不同,renice命令可以调整已运行进程的优先级,而不仅仅是启动新进程时设置优先级。

renice命令的功能:
\begin{itemize}
    \item 调整已运行进程的优先级
    \item 通过进程号(PID)指定要调整的进程
    \item 可以批量调整多个进程的优先级
    \item 支持调整指定用户或组的所有进程优先级
\end{itemize}

\begin{verbatim}
# 调整单个进程的优先级(增加nice值,降低优先级)
renice +10 1234

# 调整多个进程的优先级
renice +5 1234 5678 9012

# 降低nice值,提高进程优先级(需要root权限)
sudo renice -5 1234

# 调整指定用户的所有进程优先级
renice +10 -u username

# 调整指定组的所有进程优先级
renice +10 -g groupname

# 调整指定终端的所有进程优先级
renice +10 -t pts/0
\end{verbatim}

renice命令的常用选项:
\begin{itemize}
    \item -n:指定要设置的nice值
    \item -u:指定用户名,调整该用户的所有进程优先级
    \item -g:指定组名,调整该组的所有进程优先级
    \item -t:指定终端,调整该终端的所有进程优先级
\end{itemize}

\textbf{注意事项}:
\begin{itemize}
    \item 普通用户只能将进程的nice值增加(降低优先级),不能减少(提高优先级)
    \item root用户可以任意调整进程的nice值
    \item renice命令会影响进程的PRI值,优先级高的进程会被优先运行
    \item 可以使用"ps -l"命令查看进程的当前优先级
\end{itemize}

\subsection{作业管理命令(jobs、bg、fg)}
作业管理命令用于控制和管理shell中的作业(后台运行的进程)。在Linux系统中,当你在终端中运行命令时,可以通过这些命令来控制作业的运行状态。

\subsubsection{jobs命令}
jobs命令用于查看当前shell中运行的作业列表,包括作业号、状态和命令。

\begin{verbatim}
# 查看当前shell中的作业
jobs

# 查看详细的作业信息(包括进程组ID)
jobs -l

# 只显示运行中的作业
jobs -r

# 只显示停止的作业
jobs -s
\end{verbatim}

\subsubsection{bg命令}
bg命令用于将前台作业或停止的作业放入后台运行。

\begin{verbatim}
# 将最后一个停止的作业放入后台运行
bg

# 将指定作业号的作业放入后台运行
bg %1
\end{verbatim}

\subsubsection{fg命令}
fg命令用于将后台作业切换到前台运行。

\begin{verbatim}
# 将最后一个后台作业切换到前台运行
fg

# 将指定作业号的作业切换到前台运行
fg %1
\end{verbatim}

\textbf{作业管理操作流程示例}:
\begin{verbatim}
# 启动一个后台作业(在命令末尾添加&符号)
find . -name "*.txt" -type f | xargs grep "error" &

# 查看作业状态
jobs

# 假设作业号为1,将其切换到前台
fg %1

# 按下Ctrl+Z暂停作业
^Z

# 查看暂停的作业
jobs

# 将暂停的作业放入后台继续运行
bg %1

# 再次查看作业状态
jobs
\end{verbatim}

\textbf{作业状态说明}:
\begin{itemize}
    \item Running:作业正在运行
    \item Stopped:作业被暂停(通常是通过Ctrl+Z)
    \item Done:作业已完成
\end{itemize}

\textbf{注意事项}:
\begin{itemize}
    \item 作业管理命令只对当前shell中的作业有效
    \item 当退出shell时,后台作业可能会被终止(除非使用nohup命令)
    \item 可以使用kill命令终止后台作业,例如:kill %1
\end{itemize}

\section{服务管理}
\begin{itemize}
    \item 系统服务概念
    \item SysV init,Upstart,Systemd
    \item Systemd服务管理(systemctl)
    \item 主机名管理(hostnamectl)
    \item 服务自启动配置
    \item 日志管理(journalctl)
\end{itemize}

\subsection{hostnamectl命令}
hostnamectl是Linux系统中用于查看和修改主机名的命令,属于systemd工具集的一部分。它提供了一种统一的方式来管理系统的主机名设置,适用于使用systemd的现代Linux发行版。

\begin{verbatim}
# 查看主机名信息
hostnamectl

# 修改静态主机名(需要root权限)
sudo hostnamectl set-hostname new-hostname

# 修改Pretty主机名(用于显示的友好名称)
sudo hostnamectl set-hostname "My Server" --pretty

# 查看特定信息
hostnamectl status
hostnamectl hostname  # 只显示主机名
hostnamectl kernel    # 只显示内核版本
\end{verbatim}

hostnamectl命令的优势在于它会自动更新相关的配置文件,确保主机名的持久化设置,并且不需要重启系统即可生效。相比传统的hostname命令,它提供了更全面的主机名管理功能。

\section{磁盘管理}
\begin{itemize}
    \item 磁盘分区:fdisk,gdisk
    \item 文件系统创建:mkfs
    \item 磁盘挂载:mount,umount,/etc/fstab配置
    \item 磁盘使用情况查看:df,du
    \item 磁盘性能测试:dd,hdparm
    \item 逻辑卷管理(LVM)
\end{itemize}

\section{系统监控与性能优化}
\begin{itemize}
    \item 系统资源监控:vmstat,iostat,sar
    \item 内存管理:free,top
    \item 网络监控:netstat,ss
    \item 系统日志查看:/var/log目录,dmesg
    \item 性能瓶颈分析与优化策略
\end{itemize}

\subsection{free命令}
free命令主要用来查看系统内存、虚拟内存的大小及占用情况。

free命令的常用选项:
\begin{itemize}
    \item -b:以字节为单位显示
    \item -k:以KB为单位显示(默认)
    \item -m:以MB为单位显示
    \item -g:以GB为单位显示
    \item -h:以人类可读的格式显示(自动选择合适的单位)
    \item -t:显示总计行
    \item -s:持续显示,指定间隔秒数
    \item -c:持续显示指定次数后退出
    \item -V:显示版本信息
\end{itemize}

\begin{verbatim}
# 查看内存使用情况(默认KB单位)
free

# 以人类可读的格式查看内存使用情况
free -h

# 显示总计行
free -h -t

# 持续显示内存使用情况(每2秒一次,共10次)
free -h -s 2 -c 10

# 以MB单位查看内存使用情况
free -m
\end{verbatim}

\textbf{输出说明}:
- total:总内存大小
- used:已使用的内存大小
- free:空闲内存大小
- shared:共享内存大小
- buff/cache:缓冲区和缓存大小
- available:可用内存大小(考虑了可回收的缓存)

\textbf{注意事项}:
- available值比free值更能准确反映系统的实际可用内存
- Linux系统会使用部分内存作为缓存来提高性能,这是正常现象

\subsection{uname命令}
uname命令用于显示系统信息,包括内核版本、硬件架构、主机名等。它是一个常用的系统信息查询工具。

\begin{verbatim}
# 显示系统内核版本
uname

# 显示所有系统信息
uname -a

# 显示内核版本
uname -r

# 显示硬件架构
uname -m

# 显示主机名
uname -n

# 显示操作系统名称
uname -s

# 显示内核版本号
uname -v

# 显示处理器类型
uname -p

# 显示硬件平台
uname -i
\end{verbatim}

uname命令的常用选项:
\begin{itemize}
    \item -a:显示所有系统信息
    \item -r:显示内核版本
    \item -m:显示硬件架构
    \item -n:显示主机名
    \item -s:显示操作系统名称
    \item -v:显示内核版本号
    \item -p:显示处理器类型
    \item -i:显示硬件平台
\end{itemize}

\textbf{使用说明}:uname命令是一个简单但实用的工具,它可以快速获取系统的基本信息,特别适合在脚本中使用。通过不同的选项组合,可以获取系统的各种信息,帮助用户了解系统的配置和状态。

\textbf{示例输出}:
\begin{verbatim}
# uname -a输出示例
Linux localhost 5.4.0-100-generic x86_64 x86_64 x86_64 GNU/Linux

# 各字段含义:
# Linux:操作系统名称
# localhost:主机名
# 5.4.0-100-generic:内核版本
# x86_64:硬件架构
# x86_64:处理器类型
# x86_64:硬件平台
# GNU/Linux:操作系统类型
\end{verbatim}

\subsection{dmesg命令}
dmesg命令用实例名称和物理名称来标识连到系统上的设备。dmesg命令也用于显示系统诊断信息、操作系统版本号、物理内存大小以及其他信息。

dmesg命令的常用选项:
\begin{itemize}
    \item -c:显示信息后清除缓冲区
    \item -T:显示人类可读的时间戳
    \item -L:使用彩色输出
    \item -l:限制显示的日志级别(如err, warn, info等)
    \item -k:只显示内核消息
    \item -t:不显示时间戳
    \item -x:显示详细的内核消息
\end{itemize}

\begin{verbatim}
# 查看所有系统消息
dmesg

# 查看系统消息并显示人类可读的时间戳
dmesg -T

# 只查看错误消息
dmesg -l err

# 查看最近的系统消息(结合tail)
dmesg | tail -n 50

# 查看特定设备的相关消息
dmesg | grep -i usb

dmesg | grep -i eth

# 实时监控系统消息
dmesg -w
\end{verbatim}

\textbf{注意事项}:
- dmesg命令输出的信息可能会很多,可以使用管道和grep命令来过滤感兴趣的内容
- 对于较新的Linux系统,系统日志也可能存储在/journal或/var/log目录中
- 使用dmesg -c会清除内核环缓冲区的内容,谨慎使用

\chapter{Linux网络配置与管理}

\section{网络基础概念}
\begin{itemize}
    \item TCP/IP协议栈
    \item IP地址与子网掩码
    \item 网关与DNS
    \item 端口与服务
\end{itemize}

\section{网络配置}
\begin{itemize}
    \item 网络接口配置(ifconfig,ip命令)
    \item 静态IP配置
    \item DHCP配置
    \item 路由配置(route,ip route)
    \item DNS配置(/etc/resolv.conf)
\end{itemize}

\section{网络服务}
\begin{itemize}
    \item SSH服务(远程登录)
    \item FTP/SFTP服务(文件传输)
    \item Web服务(Apache,Nginx)
    \item DNS服务(BIND)
    \item 邮件服务(Postfix,Dovecot)
\end{itemize}

\section{防火墙配置}
\begin{itemize}
    \item iptables基础
    \item firewalld服务(新一代防火墙)
    \item 端口开放与访问控制
\end{itemize}

\chapter{Linux脚本编程}

\section{Shell脚本基础}
\begin{itemize}
    \item Shell脚本的概念
    \item 脚本文件格式与执行方式
    \item 注释与变量
    \item 输入输出(echo,read)
\end{itemize}

\section{Shell编程结构}
\begin{itemize}
    \item 条件判断(if-elif-else)
    \item 循环结构(for,while,until)
    \item 函数定义与调用
    \item 数组与字符串处理
\end{itemize}

\section{脚本调试与优化}
\begin{itemize}
    \item 脚本调试技巧
    \item 错误处理机制
    \item 性能优化建议
    \item 脚本安全考虑
\end{itemize}

\section{实用脚本示例}
\begin{itemize}
    \item 系统监控脚本
    \item 备份脚本
    \item 自动化部署脚本
    \item 日志分析脚本
\end{itemize}

\chapter{Linux高级应用}

\section{虚拟化技术}
\begin{itemize}
    \item KVM虚拟化
    \item Docker容器技术
    \item 容器编排(Kubernetes)
\end{itemize}

\section{存储管理}
\begin{itemize}
    \item RAID技术
    \item NFS网络文件系统
    \item Samba服务
    \item iSCSI存储
\end{itemize}

\section{安全加固}
\begin{itemize}
    \item 系统安全基线配置
    \item SSH安全配置
    \item 入侵检测与防御
    \item 安全审计
\end{itemize}

\chapter{Linux学习资源与进阶建议}

\section{推荐书籍}
\section{在线教程与文档}
\section{社区与论坛}
\section{实践项目建议}
\section{认证考试(RHCE,LPIC等)}

\vspace{1cm}
\hrule
\vspace{0.5cm}
\begin{center}
    \textit{注:此提纲仅作为学习Linux的框架,具体内容将在后续逐步补充。}
\end{center}

\end{document}
