% MySQL
% 模板参考:D:\Git\Book\0_book\book.tex

\documentclass[12pt,UTF8]{ctexbook}

% 设置纸张信息。
\input{../../../../Book/common/page}

% 设置字体,并解决显示难检字问题。
\xeCJKsetup{AutoFallBack=true}
% 注意:ctexbook类已默认设置SimSun为CJKrmdefault
% 以下设置用于确保字体回退和扩展字体可用
% 仅设置扩展字体以避免与默认设置冲突
\setCJKfamilyfont{hei}{SimHei}
\setCJKfamilyfont{kai}{KaiTi}

% 目录 chapter 级别加点(.)。
\usepackage{titletoc}
\titlecontents{chapter}[0pt]{\vspace{3mm}\bf\addvspace{2pt}\filright}{\contentspush{\thecontentslabel\hspace{0.8em}}}{}{\titlerule*[8pt]{.}\contentspage}

% 支持目录点击跳转
\usepackage[colorlinks,linkcolor=blue,citecolor=blue,urlcolor=blue]{hyperref}

% 设置 part 和 chapter 标题格式。
\ctexset{
	part/name= {第,卷},
	part/number={\chinese{part}},
	chapter/name={第,章},
	chapter/number={\arabic{chapter}}
}

% 图片相关设置。
\usepackage{graphicx}
\graphicspath{{Images/}}

% 设置署名格式。
\newenvironment{shuming}{\hfill\zihao{4}}

% 注脚每页重新编号,避免编号过大。
\usepackage[perpage]{footmisc}

% 设置古文原文格式。
\newenvironment{yuanwen}{\bfseries\zihao{4}}

% 列表项向右偏移。
\usepackage{enumitem}

\title{\heiti\zihao{0} MySQL}
\author{WangFei}
\date{\today}

\begin{document}

\maketitle
\tableofcontents

\mainmatter

\chapter{备份与还原}
\section{还原备份慢的解决方法}

\subsection{问题现象}
使用\verb|mysql -uroot -p < dump.sql|或\verb|source|命令导入十分耗时。

\subsection{原因分析}
可能是使用Navicat导出,可以打开导出文件看到,Navicat导出的都是一条一条insert语句,而非批量语句。

\subsection{解决方案}

\subsubsection{使用mysqldump导出}
默认导出的语句是批量插入语句。

默认批量插入的条数是250条,可使用\verb|--extended-insert=10000|指定批量插入时的数据:
\begin{verbatim}
# 基本导出命令
mysqlpump -u 用户名 -p 数据库名 > backup.sql

# 指定批量插入条数
mysqlpump -u 用户名 -p --extended-insert=10000 数据库名 > backup.sql
\end{verbatim}

\subsubsection{其他可能的原因及解决方案}
\begin{itemize}[leftmargin=2cm]
    \item 网络带宽限制:如果是远程导入,检查网络连接
    \item 服务器配置:调整\verb|innodb_buffer_pool_size|、\verb|innodb_log_file_size|等参数
    \item 磁盘I/O瓶颈:使用SSD存储或优化磁盘RAID配置
    \item 表结构问题:检查是否有过多的索引或触发器
\end{itemize}

\backmatter

\end{document}